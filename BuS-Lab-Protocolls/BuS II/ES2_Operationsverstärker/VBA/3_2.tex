\begin{figure}[H]
  \begin{center}
    \includegraphics[width=0.618\textwidth]{1_2/opv_nofeed.pdf}
    \end{center}
    \caption{OPV-Kennlinie ohne Rückkopplung (ideal)}
 \end{figure}
\begin{figure}[H]
  \begin{center}
    \includegraphics[width=0.618\textwidth]{1_2/opv_feed.pdf}
    \end{center}
    \caption{OPV-Kennlinie mit Rückkopplung}
 \end{figure}

Durch Rückführung eines Teils des Ausgangs- auf das Eingangssignal durch ein
Rückkopplungsnetzwerk wird der Operationsverstärker in einen linearen
Arbeitsbereich gebracht, wodurch die Verstärkung nicht mehr den Wert der der Leerlaufverstärkung (Abb. 1), sondern einen kontrollierten Verstärkungswert (Abb. 2) annimmt.
