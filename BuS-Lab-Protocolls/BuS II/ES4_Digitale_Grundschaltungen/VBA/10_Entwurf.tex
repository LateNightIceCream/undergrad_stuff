Zum Zählerentwurf muss eine Wahrheitstabelle mit den jeweiligen \glqq
momentanen\grqq Zählerzuständen (Ausgangsbits 0...n) und den gewünschten
Folgezuständen erstellt werden (je nach Zählart). Für jeden
Folge-Ausgangszustand müssen dann die minimierten Schaltfunktionen erstellt
werden, wobei darauf zu achten ist, dass die Variable des momentan behandelten
Zustandes nicht herausoptimiert wird und somit in jedem Einzelterm vorhanden ist.

Durch gleichsetzen der sich ergebenden minimierten Schaltfunktion mit der
Schaltfunktion des verwendeten Flipflops kann für jeden Flipflopeingang über
einen Koeffizientenvergleich die Eingangsschaltfunktion bestimmt werden.