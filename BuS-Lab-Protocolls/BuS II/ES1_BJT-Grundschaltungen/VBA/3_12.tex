Das Millertheorem besagt, dass sich eine zwischen Ein- und Ausgangskreis
befindliche Impedanz durch zwei separate, jeweils im Ein- und Ausgangskreis
befindliche, Impedanzen ersetzen lässt. Dies ist bei der Analyse des
Kleinsignalersatzschaltbildes mit parasitären Kapazitäten (Abb. 2) hilfreich, da dort die
Sperrschichtkapazität $C_{BC}$ den Ausgangskreis mit dem Eingangskreis koppelt.
Sie lässt sich in zwei resultierende Miller-Kapazitäten transformieren.
\[I_1 = \frac{U_1 - U_2}{Z} = \frac{U_1 - V_u \cdot U_1}{Z} = \frac{U_1
    (1-V_u)}{Z} = \frac{U_1}{ \underbrace{\frac{Z}{1-V_u}}_{Z'} }\]
\[Z' = \frac{Z}{1-V_u}\]
\noindent analog gilt für die ausgangsseitige Impedanz
\[Z'' = \frac{Z}{1- \frac{1}{V_u} }\]

\noindent setzt man
\[Z = \frac{1}{j \omega C_{BC}}\]
\noindent für die Sperrschichtmillerkapazitäten, erhält man
\[C_{BC}' = C_{BC}(1-V_u)\]
\[C_{BC}'' = C_{BC}(1-\frac{1}{V_u})\]
