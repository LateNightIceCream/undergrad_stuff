\documentclass[a4paper, 12pt]{article}

%%%  MIKROCONTROLLERTECHNIK PREAMBLE
%%% 2020
%%%%%%%%%%%%%%%%%%%%%%%%%%%%%%%


%%% PACKAGES
%%%%%%%%%%%%%%%%%%%%%%%%%%%

\usepackage[ngerman]{babel}

\usepackage[utf8]{inputenc}
\usepackage{amsmath}
\usepackage{pgfplots}
\usepackage{tikz}
\usepackage[many]{tcolorbox}
\usepackage{graphicx}
\graphicspath{ {./graphics/} }
\usepackage{pdfpages}
\usepackage{dashrule}
\usepackage{float}
\usepackage{siunitx}
\usepackage{trfsigns}
\usepackage{booktabs}
\usepackage[european]{circuitikz}
\usepackage{listings}
\usepackage{titlesec}

\usepackage{fontspec}
\usepackage{tgheros}
\usepackage{tgcursor}

\usepackage{sansmath}
\sansmath


%%% DOCUMENT GEOMETRY
%%%%%%%%%%%%%%%%%%%%%%%%%%%

\usepackage{geometry}
\geometry{
 a4paper,
 total={0.7639320225002104\paperwidth,0.6180339887498948\paperheight},
 top = 0.1458980337503154\paperheight,
 bottom = 0.1458980337503154\paperheight
 }
\setlength{\jot}{0.013155617496424828\paperheight}
\linespread{1.1458980337503154}

\setlength{\parskip}{0.013155617496424828\paperheight} % paragraph spacing

\titlespacing*{\section}
{0pt}{\smallvert}{0.618\smallvert}
\titlespacing*{\subsection}
{0pt}{0.618\smallvert}{0.382\smallvert}
%%% FONTS
%%%%%%%%%%%%%%%%%%%%%%%%%%%

\setmainfont{TeX Gyre Heros}
\setmonofont{Inconsolatazi4}

%%% COLORS
%%%%%%%%%%%%%%%%%%%%%%%%%%%

%%% LATEX COLORS (OPENCOLORS)
%%% 2020
%%%%%%%%%%%%%%%%%%%%%%%%%%%%%%%

% GRAY
\definecolor{gray0}{HTML}{f8f9fa}
\definecolor{gray1}{HTML}{f1f3f5}
\definecolor{gray2}{HTML}{e9ecef}
\definecolor{gray3}{HTML}{dee2e6}
\definecolor{gray4}{HTML}{ced4da}
\definecolor{gray5}{HTML}{adb5bd}
\definecolor{gray6}{HTML}{868e96}
\definecolor{gray7}{HTML}{495057}
\definecolor{gray8}{HTML}{343a40}
\definecolor{gray9}{HTML}{212529}

% RED
\definecolor{red0}{HTML}{fff5f5}
\definecolor{red1}{HTML}{ffe3e3}
\definecolor{red2}{HTML}{ffc9c9}
\definecolor{red3}{HTML}{ffa8a8}
\definecolor{red4}{HTML}{ff8787}
\definecolor{red5}{HTML}{ff6b6b}
\definecolor{red6}{HTML}{fa5252}
\definecolor{red7}{HTML}{f03e3e}
\definecolor{red8}{HTML}{e03131}
\definecolor{red9}{HTML}{c92a2a}

% PINK
\definecolor{pink0}{HTML}{fff0f6}
\definecolor{pink1}{HTML}{ffdeeb}
\definecolor{pink2}{HTML}{fcc2d7}
\definecolor{pink3}{HTML}{faa2c1}
\definecolor{pink4}{HTML}{f783ac}
\definecolor{pink5}{HTML}{f06595}
\definecolor{pink6}{HTML}{e64980}
\definecolor{pink7}{HTML}{d6336c}
\definecolor{pink8}{HTML}{c2255c}
\definecolor{pink9}{HTML}{a61e4d}

% GRAPE
\definecolor{grape0}{HTML}{f8f0fc}
\definecolor{grape1}{HTML}{f3d9fa}
\definecolor{grape2}{HTML}{eebefa}
\definecolor{grape3}{HTML}{e599f7}
\definecolor{grape4}{HTML}{da77f2}
\definecolor{grape5}{HTML}{cc5de8}
\definecolor{grape6}{HTML}{be4bdb}
\definecolor{grape7}{HTML}{ae3ec9}
\definecolor{grape8}{HTML}{9c36b5}
\definecolor{grape9}{HTML}{862e9c}

% VIOLET
\definecolor{violet0}{HTML}{f3f0ff}
\definecolor{violet1}{HTML}{e5dbff}
\definecolor{violet2}{HTML}{d0bfff}
\definecolor{violet3}{HTML}{b197fc}
\definecolor{violet4}{HTML}{9775fa}
\definecolor{violet5}{HTML}{845ef7}
\definecolor{violet6}{HTML}{7950f2}
\definecolor{violet7}{HTML}{7048e8}
\definecolor{violet8}{HTML}{6741d9}
\definecolor{violet9}{HTML}{5f3dc4}

% INDIGO
\definecolor{indigo0}{HTML}{edf2ff}
\definecolor{indigo1}{HTML}{dbe4ff}
\definecolor{indigo2}{HTML}{bac8ff}
\definecolor{indigo3}{HTML}{91a7ff}
\definecolor{indigo4}{HTML}{748ffc}
\definecolor{indigo5}{HTML}{5c7cfa}
\definecolor{indigo6}{HTML}{4c6ef5}
\definecolor{indigo7}{HTML}{4263eb}
\definecolor{indigo8}{HTML}{3b5bdb}
\definecolor{indigo9}{HTML}{364fc7}

% BLUE
\definecolor{blue0}{HTML}{e7f5ff}
\definecolor{blue1}{HTML}{d0ebff}
\definecolor{blue2}{HTML}{a5d8ff}
\definecolor{blue3}{HTML}{74c0fc}
\definecolor{blue4}{HTML}{4dabf7}
\definecolor{blue5}{HTML}{339af0}
\definecolor{blue6}{HTML}{228be6}
\definecolor{blue7}{HTML}{1c7ed6}
\definecolor{blue8}{HTML}{1971c2}
\definecolor{blue9}{HTML}{1864ab}

% CYAN
\definecolor{cyan0}{HTML}{e3fafc}
\definecolor{cyan1}{HTML}{c5f6fa}
\definecolor{cyan2}{HTML}{99e9f2}
\definecolor{cyan3}{HTML}{66d9e8}
\definecolor{cyan4}{HTML}{3bc9db}
\definecolor{cyan5}{HTML}{22b8cf}
\definecolor{cyan6}{HTML}{15aabf}
\definecolor{cyan7}{HTML}{1098ad}
\definecolor{cyan8}{HTML}{0c8599}
\definecolor{cyan9}{HTML}{0b7285}

% TEAL 
\definecolor{teal0}{HTML}{e6fcf5}
\definecolor{teal1}{HTML}{c3fae8}
\definecolor{teal2}{HTML}{96f2d7}
\definecolor{teal3}{HTML}{63e6be}
\definecolor{teal4}{HTML}{38d9a9}
\definecolor{teal5}{HTML}{20c997}
\definecolor{teal6}{HTML}{12b886}
\definecolor{teal7}{HTML}{0ca678}
\definecolor{teal8}{HTML}{099268}
\definecolor{teal9}{HTML}{087f5b}

% GREEN
\definecolor{green0}{HTML}{ebfbee}
\definecolor{green1}{HTML}{d3f9d8}
\definecolor{green2}{HTML}{b2f2bb}
\definecolor{green3}{HTML}{8ce99a}
\definecolor{green4}{HTML}{69db7c}
\definecolor{green5}{HTML}{51cf66}
\definecolor{green6}{HTML}{40c057}
\definecolor{green7}{HTML}{37b24d}
\definecolor{green8}{HTML}{2f9e44}
\definecolor{green9}{HTML}{2b8a3e}

% LIME
\definecolor{lime0}{HTML}{f4fce3}
\definecolor{lime1}{HTML}{e9fac8}
\definecolor{lime2}{HTML}{d8f5a2}
\definecolor{lime3}{HTML}{c0eb75}
\definecolor{lime4}{HTML}{a9e34b}
\definecolor{lime5}{HTML}{94d82d}
\definecolor{lime6}{HTML}{82c91e}
\definecolor{lime7}{HTML}{74b816}
\definecolor{lime8}{HTML}{66a80f}
\definecolor{lime9}{HTML}{5c940d}

% YELLOW
\definecolor{yellow0}{HTML}{fff9db}
\definecolor{yellow1}{HTML}{fff3bf}
\definecolor{yellow2}{HTML}{ffec99}
\definecolor{yellow3}{HTML}{ffe066}
\definecolor{yellow4}{HTML}{ffd43b}
\definecolor{yellow5}{HTML}{fcc419}
\definecolor{yellow6}{HTML}{fab005}
\definecolor{yellow7}{HTML}{f59f00}
\definecolor{yellow8}{HTML}{f08c00}
\definecolor{yellow9}{HTML}{e67700}

% ORANGE
\definecolor{orange0}{HTML}{fff4e6}
\definecolor{orange1}{HTML}{ffe8cc}
\definecolor{orange2}{HTML}{ffd8a8}
\definecolor{orange3}{HTML}{ffc078}
\definecolor{orange4}{HTML}{ffa94d}
\definecolor{orange5}{HTML}{ff922b}
\definecolor{orange6}{HTML}{fd7e14}
\definecolor{orange7}{HTML}{f76707}
\definecolor{orange8}{HTML}{e8590c}
\definecolor{orange9}{HTML}{d9480f}

%\definecolor{red1}{HTML}{f38181}
%\definecolor{yellow1}{HTML}{fce38a}
%\definecolor{green1}{HTML}{95e1d3}
%\definecolor{blue1}{HTML}{66bfbf}
\definecolor{hsblue}{HTML}{00b1db}
\definecolor{hsgrey}{HTML}{afafaf}

\usepackage[%  
    colorlinks=true,
    pdfborder={0 0 0},
    linkcolor=gray6
]{hyperref}

%%% CONSTANTS
%%%%%%%%%%%%%%%%%%%%%%%%%%%
\newlength{\smallvert}
\setlength{\smallvert}{0.0131556\paperheight}

%%% PACKAGE SETTINGS
%%%%%%%%%%%%%%%%%%%%%%%%%%%

% Code listing
\lstset{
  language=C,
  basicstyle=\ttfamily,
  keywordstyle=\bfseries\color{hsblue},
  identifierstyle=\color{black},
  rulecolor=\color{gray1},
  frame=single,
  frameround=tttt,
  %framesep=1.382em,
  backgroundcolor=\color{gray0}
}

%%% COMMANDS
%%%%%%%%%%%%%%%%%%%%%%%%%%%

% differential d
\newcommand*\dif{\mathop{}\!\mathrm{d}}

% horizontal line
\newcommand{\holine}[1]{
  	\begin{center}
	  	\noindent{\color{hsgrey}\hdashrule[0ex]{#1}{1pt}{3mm}}\\%[0.0131556\paperheight]
  	\end{center}
}

% mini section
\newcommand{\minisec}[1]{ \noindent\underline{\textit {#1} } \\}

% quick function plot
\newcommand{\plotfun}[3]{
  \vspace{0.021286\paperheight}
  \begin{center}
    \begin{tikzpicture}
      \begin{axis}[
        axis x line=center,
        axis y line=center,
        ]
        \addplot[draw=red1][domain=#2:#3]{#1};
      \end{axis}
    \end{tikzpicture}
  \end{center}
}

% box for notes
\newcommand{\notebox}[1]{

\tcbset{colback=white,colframe=red1!100!black,title=Note!,width=0.618\paperwidth,arc=0pt}

 \begin{center}
  \begin{tcolorbox}[]
   #1 
  \end{tcolorbox}
 
 \end{center} 
 
}

% box for equation
\newcommand{\eqbox}[2]{
	
	\tcbset{colback=white,colframe=hsblue!100!black,title=,width=#2,arc=0pt}
	
	\begin{center}
		\begin{tcolorbox}[ams align*]
				#1
		\end{tcolorbox}
		
	\end{center} 
	
}

\newcommand{\inlinecode}[1] {
  \inlinebox{\lstinline[language=C, identifierstyle=\color{gray7}, basicstyle=\color{gray7}\ttfamily, keywordstyle=\color{gray7}]{#1}}
}

\newtcbox{\inlinebox}{enhanced,nobeforeafter,tcbox raise base,boxrule=0pt,top=0.062em,bottom=0.062em,
  right=0.382em,left=0.382em,arc=0.382em,boxsep=0.1em,before upper={\vphantom{dlg}},
  colframe=white,colback=gray0}

\newcommand{\commentGray}[1] {
  {\color{gray4}{#1}}
}

% END OF PREAMBLE

\setlength\columnsep{0.145898\paperwidth}
%%%%%%%%%%%%%%%%%%%%%%%%%%%%%%%%%%%%%

\begin{document}

%%%%%%%%%%%%%%%%%%%%%%%%%%%%%%%%%%%%%
  \includepdf{./titlepage/titlepage.pdf}
  \clearpage
  \setcounter{page}{1}
%%%%%%%%%%%%%%%%%%%%%%%%%%%%%%%%%%%%%

\section{Vorbereitungsaufgaben}

% 1.2
\subsection{Aufbau und Wirkungsweise eines pn-Übergangs}
\begin{center}
  \includegraphics[width=\textwidth]{1_1/pn}
\end{center}

\noindent Dotiert man einen Halbleiterkristall (z.B. \ce{Si} oder \ce{Ge}) mit
Fremdatomen, wird die elektrische Leitfähigkeit des Halbleiters beeinflusst.

\begin{itemize}
\item{
    Bei Dotierung mit
    5-(oder höher-)wertigen Atomen (z.B. \ce{P} oder \ce{As}) geraten zusätzliche
    Elektronen in das Leitungsband des Halbleiterkristalls; es wird \emph{n-Leitung}
    provoziert
  }

\item{
    Bei Dotierung mit
    3-(oder geringer-)wertigen Atomen (z.B. \ce{B} oder \ce{Ga}) entstehen
    Elektronenfehlstellen im Valenzband des Halbleiterkristalls; es wird \emph{p-Leitung}
    provoziert.
  }
\end{itemize}

Die Elektronenkonzentration im n-Leiter ist somit höher als die im p-Leiter.
Bringt man unterschiedlich dotierte Halbleiter in Kontakt, kommt es durch
Diffusion zum Übergang von (höher-energetischen) Elektronen im Leitungsband des
n-Leiters in das (nieder-energetische) Valenzband des p-Leiters. Im p-Leiter
werden dann die Elektronenfehlstellen gefüllt und es entstehen negative Ionen;
Im n-Leiter werden durch die Elektronenwanderung Fehlstellen
von den Elektronen zurückgelassen, wodurch sich dort positive Ionen bilden. Die
Diffusion findet so lange statt, bis die Ladung bzw. das elektrische Feld der gebildeten Ionen einem weiteren
Elektronenübergang vollständig entgegenwirkt. Die verbleibende Übergangszone,
die den Ladungstransfer verhindert
wird folglich \emph{Raumladungszone} genannt. 

\begin{center}
  \includegraphics[width=\textwidth]{1_1/pn2}
\end{center}

% 1.1
\subsection{Aufbau und Wirkungsweise einer Diode}

\begin{center}
\begin{circuitikz}
  \draw (0,0) to[Do] (2,0);
\end{circuitikz}
\end{center}

\noindent Die elektrischen Eigenschaften des pn-Übergangs können technisch ausgenutzt werden, um eine \emph{Diode} zu realisieren. Der Stromfluss durch den pn-Übergang ist von der Polarität der über ihn angelegten Spannung abhängig. Man definiert daher die Orientierung der Diode in \emph{Sperr-} und \emph{Durchlassrichtung}.

Will man einen Strom durch die Diode treiben, so müssen die Elektronen des
n-Gebiets bzw. die Fehlstellen des p-Gebiets die Raumladungszone überqueren können.
Legt man eine negative Spannung über den pn-Übergang/die Diode, das heißt positives Potential an den n- und
negatives an den p-Leiter, wirkt die Influenz des äußeren elektrischen Feldes
so, dass sich die Majoritätsladungsträger des jeweiligen Gebiets (Elektronen im n-
und 'Fehlstellen' im p-Gebiet) von der Raumladungszone entfernen und diese somit
vergrößern\footnote[1]{Fehlstellen bewegen sich nicht tatsächlich, sondern nur modellhaft.
Sie stellen die positiv geladenen, \emph{ortsfesten} Atomrümpfe dar, die u.a. durch die
Elektronenbewegung hinterlassen werden.}. Die Diode nimmt einen statischen Zustand (bezüglich der
Majoritätsladungsträger) ein und wirkt somit elektrisch isolierend/sperrend. Reale Dioden besitzen allerdings eine
maximale Sperrspannung, ab welcher sie durchbrechen und ihre
Sperreigenschaft verlieren.


\begin{center}
  \includegraphics[width=\textwidth]{1_2/diode1}
\end{center}

Legt man eine (ausreichend) positive Spannung über die Diode, also positives Potential an den
p- und negatives an den n-Leiter, bewegen sich die Majoritätsladungsträger des
jeweiligen Bereichs in Richtung der Raumladungszone und verkleinern diese
dadurch. In dieser Richtung wirkt die Diode elektrisch leitfähig.

\begin{center}
  \includegraphics[width=\textwidth]{1_2/diode2}
\end{center}


% 1.3
\subsection{Z- und Schottky-Dioden}

% Z-Dioden
\begin{center}
\begin{circuitikz}
  \draw (0,0) to[Zener diode] (2,0);
\end{circuitikz}
\end{center}

Die Z-Diode ist eine besondere Diodenbauform, die kontrolliert im Durchbruchbereich
arbeiten kann. In Sperrrichtung betrieben arbeitet sie bis zu einer bestimmten
\emph{Z-Spannung}, ab welcher sie durchbricht und ihre Leitfähigkeit
exponentiell steigt. Im Durchlassbereich verhält sich die Z-Diode dagegen wie
eine normale Diode.
Z-Dioden eignen sich zur Realisierung von Spannungsstabilisierungsschaltungen (1.4).

\holine{\textwidth}

% Schottky-Dioden
\begin{center}
\begin{circuitikz}
  \draw (0,0) to[Schottky diode] (2,0);
\end{circuitikz}
\end{center}

Schottky-Dioden haben keinen üblichen Halbleiter-Halbleiter-, sondern einen Metall-Halbleiter-Übergang.
Charakteristisch sind niedrige Durchlassspanunngen im Bereich von $150 - 450 \si{\milli\volt}$.



% 1.4
\subsection{Spannungsstabilisierungsschaltung}

\begin{center}
\begin{circuitikz}
  \draw (0, 0) to[V=$U_e$] (0,-3);
  \draw (0, 0) to[R=$R_V$] (4,0);
  \draw (4,-3) to[Zener diode, *-*] (4,0);
  \draw (4,0) to[short] (6,0);
  \draw (6,0) to[R=$R_L$] (6,-3);
  \draw (0,-3) to[short] (6,-3);
\end{circuitikz}
\end{center}

Die Spannungsstabilisierungsschaltung dient dazu, eine konstante Ausgangsspannung über
einer Last auch bei schwankender Eingangsspannung ($>=U_Z$) zu gewährleisten.
Geht man davon aus, dass die Spannung über der Z-Diode nach erreichen der
Z-Spannung mit steigendem Strom konstant bleibt, so hält die Diode auch durch
ihre Parallelschaltung mit der Last die Spannung über diese konstant, indem sie
den Laststrom konstant hält. Mit steigender Eingangsspannung/steigendem
Eingangsstrom steigt daher der Strom durch die Z-Diode, während der Strom durch
die Last unverändert bleibt.

Bei Einsatz der Schaltung müssen Vor- und Lastwiderstand so dimensioniert werden, dass der minimale Strom,
den die Z-Diode benötigt, um ihre Regulationsfunktion zu bieten, und der
maximale Strom, der aufgrund von Wärmeentwicklung nicht überschritten werden
darf, eingehalten werden.

Über Messung der Ausgangsspannung und Bezug dieser auf die Eingangsspannung kann
die Schaltung getestet werden.

Die Spannungsstabilisierungsschaltung ist allerdings gerade bei hohen Strömen nicht besonders effizient,
da jede in der Last unerwünschte Leistung in der Z-Diode bzw. dem Vorwiderstand
in Wärme umgesetzt wird.

% 1.5
\subsection{Dimensionierung des Vorwiderstands}

\begin{align*}
  U_{e_{\textrm{min}}} &= 8.5 \,\ \si{\volt}\\
  U_{e_{\textrm{max}}} &= 12.0 \,\ \si{\volt}\\
  I_{L_{\textrm{min}}} &= 50 \,\ \si{\milli\ampere}\\
  I_{L_{\textrm{max}}} &= 130 \,\ \si{\milli\ampere}\\
  U_{Z} &= 5.6 \,\ \si{\volt}\\
  I_{Z_{\textrm{min}}}/I_{Z_{\textrm{max}}} &= 20 \,\ \si{\milli\ampere} / 300 \,\ \si{\milli\ampere}\\
\end{align*}

\begin{gather*}
  R_{V_{\textrm{max}}} = \frac{U_{e_{\textrm{min}}} -
    U_Z}{I_{Z_{\textrm{min}}}+I_{L_{\textrm{max}}}}=\frac{8.5 \si{\volt} - 5.6
    \si{\volt}}{20 \si{\milli\ampere} + 130 \si{\milli\ampere}} = 19.33 \,\ \si{\ohm}\\[\smallvert]
  R_{V_{\textrm{min}}} = \frac{U_{e_{\textrm{max}}} -
    U_Z}{I_{Z_{\textrm{max}}}+I_{L_{\textrm{min}}}} = \frac{12 \si{\volt} -
    5.6\si{\volt}}{300 \si{\milli\ampere} + 50 \si{\milli\ampere}} = 18.29 \,\
  \si{\ohm}
\end{gather*}

$\rightarrow$ z.B. ein $18.7 \,\ \si{\ohm}$ Widerstand aus der E48-Reihe.

% 1.6
\subsection{Diodenkennlinie}

\begin{center}
  \includegraphics[width=\textwidth]{1_6/diodenkennline}
\end{center}

$$ I(U) = I_s \cdot (e^{\frac{U}{m \cdot U_T} }-1)$$


% 1.7
\subsection{Diodengrenzwerte}

Der Strom, der durch die Diode fließt und die Spannung, die über der Diode
anliegt ergeben im Produkt die Leistung, die die Diode in (z.B.) Wärme umsetzt.
Überschreitet die Leistung einen Grenzwert, kann es zur Überhitzung und u.U.
Zerstörung der Diode kommen. Grafisch kann man dies durch den Schnittpunkt der
Diodenkennlinie mit einer
\emph{Verlustleistungshyperbel} darstellen, welche alle Wertepaare von Strom und
Spannung abbildet, die die maximale Verlustleistung der Diode bilden.

Wie in 1.2 bereits erwähnt, sperren Dioden nicht bis zu beliebigen Spannungen
sondern brechen bei einer maximalen Sperrspannung durch, wodurch weiteres
Sperren verhindert wird.


% 2
%%%%%%%%%%%%%%%%%%%%%%%%%%%%%%%%%%%%%
\pagebreak
\includepdf{./titlepage/titlepage2.pdf}
  \clearpage
\setcounter{page}{1}

\setcounter{section}{0}
\section{Versuchsaufgaben}

% 2.1

\subsection{Diodenkennline BY500}
\begin{table}[H]
\begin{center}
\begin{tabular}{@{}c|c@{}}
\toprule
$U / \si{\volt}$    & $I / \si{\milli\ampere}$     \\ \midrule
0    & 0     \\
0.5  & 2.9   \\
0.6  & 15.8  \\
0.65 & 34.3  \\
0.67 & 46.6  \\
0.69 & 63    \\
0.7  & 72.2  \\
0.71 & 84.3  \\
0.72 & 100.7 \\
0.73 & 118   \\
0.8  & 336.4 \\
0.84 & 650   \\
0.87 & 1000  \\ \bottomrule
\end{tabular}
\end{center}
\caption{Messwerte der Aufgabe 3.1, Diode in Durchlassrichtung}
\end{table}

\begin{table}[H]
  \begin{center}
\begin{tabular}{@{}c|c@{}}
\toprule
$U / \si{\volt}$  & $I / \si{\micro\ampere}$    \\ \midrule
-0  & -0    \\
-5  & -0.06 \\
-10 & -0.07 \\
-15 & -0.08 \\
-20 & -0.09 \\
-25 & -0.1  \\
-30 & -0.1  \\ \bottomrule
\end{tabular}
  \end{center}
\caption{Messwerte der Aufgabe 3.1, Diode in Sperrrichtung}
\end{table}

\begin{center}
 \includegraphics[width=\textwidth]{2_1/2_1_kennlinie}
\end{center}

\begin{center}
 \includegraphics[width=\textwidth]{2_1/2_1_kennlinie_sperr}
\end{center}

% 2.2
\pagebreak
\subsection{Diodenkennline ZY5,6}

\begin{multicols}{2}
\begin{center}
  \begin{table}[H]
\begin{center}
\begin{tabular}{@{}c|c@{}}
\toprule
$U / \si{\volt}$     & $I / \si{\milli\ampere}$     \\ \midrule
0.5   & 0.13  \\
0.6   & 0.819 \\
0.65  & 3.06  \\
0.7   & 14.23 \\
0.71  & 16.5  \\
0.72  & 22.22 \\
0.73  & 32.5  \\
0.74  & 44.43 \\
0.75  & 63.6  \\
0.76  & 134   \\
0.77  & 166   \\
0.775 & 194   \\
0.777 & 237   \\ \bottomrule
\end{tabular}
\end{center}
\caption{Messwerte der Aufgabe 3.2, Z-Diode in Durchlassrichtung}
\end{table}
\columnbreak

  \begin{table}[H]
\begin{center}
\begin{tabular}{@{}c|c@{}}
\toprule
$U / \si{\volt}$     & $I / \si{\milli\ampere}$        \\ \midrule
0     & 0        \\
-0.5  & -0.0002  \\
-1    & -0.00096 \\
-1.5  & -0.0032  \\
-2    & -0.008   \\
-2.5  & -0.0178  \\
-3    & -0.0352  \\
-3.5  & -0.07    \\
-4    & -0.124   \\
-4.5  & -0.252   \\
-5    & -0.61    \\
-5.5  & -2.365   \\
-5.66 & -20      \\
-5.68 & -37      \\
-5.75 & -66.7    \\
-5.82 & -146     \\
-5.85 & -202     \\
-5.96 & -297     \\ \bottomrule
\end{tabular}
\end{center}
\caption{Messwerte der Aufgabe 3.2, Z-Diode in Sperrrichtung}
\end{table}

\end{center}
\end{multicols}



\begin{center}
 \includegraphics[width=\textwidth]{2_2/2_2_kennlinie}
\end{center}

Aus den Messwerten lassen sich die charakteristischen Eigenschaften einer
Z-Diode gut erkennen.

% 2.3
\subsection{Z-Spannung und differentieller Widerstand}

Zur Ermittlung des differentiellen Widerstands $r_z$ wurden die letzten beiden
Wertepaare der Messreihe verwendet, da diese die Linearisierung (theoretisch) am besten
abbilden.

$$ r_z = \frac{\Delta U}{\Delta I} $$

\begin{gather*}
  U_2 = -5.96 \,\ \si{\volt}\\
  I_2 = -297 \,\ \si{\milli\ampere}\\
  U_1 = -5.85 \,\ \si{\volt}\\
  I_1 = -202 \,\ \si{\milli\ampere}\\
  r_z = \frac{U_2 - U_1}{I_2-I_1} = \frac{-5.96 - (-5.85)}{-0.297 - (-0.202)}
  \frac{\si{\volt}}{\si{\ampere}} = 1.158 \,\ \si{\ohm}
\end{gather*}

Mit dem differentiellen Widerstand konnte dann die Z-Spannung $U_z$ über die Geradengleichung

$$U(I) = I \cdot r_z + U_z$$

\noindent bestimmt werden.

\noindent Nach Einsetzen eines der Wertepaare erhält man:
$$ U_z = U(I) - I\cdot r_z = -5.96 \si{\volt} - (-0.297 \si{\ampere} \cdot 1.158
\si{\ohm}) = -5.616 \,\ \si{\volt} $$

\noindent Dieser Wert passt zu der im Datenblatt der Diode angegebenen
Z-Spannung von $U_{Z_{DB}} = 5.6 \,\ \si{\volt}$.
\begin{center}
 \includegraphics[width=\textwidth]{2_3/2_3_widerstand}
\end{center}

% 2.4
\subsection{Spannungsstabilisierung bei veränderlicher Eingangsspannung}

\begin{table}[H]
\begin{center}
\begin{tabular}{c|c}
\toprule
$U_e / \si{\volt} $ & $U_a / \si{\volt}$ \\\midrule
8.5                                               & 0.772                                             \\
9                                                 & 0.775                                             \\
9.5                                               & 0.7768                                            \\
10                                                & 0.7775                                            \\
10.5                                              & 0.778                                             \\
11                                                & 0.7786                                            \\
11.5                                              & 0.7787                                            \\
12                                                & 0.7788                 \\\bottomrule 
\end{tabular}
\end{center}
\caption{Messwerte der Aufgabe 3.5}
\end{table}

$$ \Delta U_a = U_{a_{max}} - U_{a_{min}} = (0.7788 - 0.772) \si{\volt} =
0.0068 \,\ \si{\volt}$$
$$ \Delta U_e = U_{e_{max}} - U_{e_{min}} = (12 - 8.5) \si{\volt} = 3.5 \,\ \si{\volt}$$\\
$$S = \frac{\Delta U_a}{\Delta U_e} = 0.001943 = 0.1943 \,\ \si{\percent}$$

\begin{center}
 \includegraphics[width=\textwidth]{2_4/2_4_kennlinie}
\end{center}


% 2.5
\subsection{Spannungsstabilisierung bei veränderlichem Laststrom}
\begin{table}[H]
\begin{center}
\begin{tabular}{@{}c|c|c@{}}
\toprule
$R_L / \si{\ohm}$   & $U_a / \si{\volt}$ & $I_L / \si{\ampere}$                   \\ \midrule
40  & 5.79    & 0.145              \\
60  & 5.82    & 0.097                \\
80  & 5.84    & 0.073                \\
100 & 5.86    & 0.059 \\
120 & 5.87    & 0.049 \\ \bottomrule
\end{tabular}
\end{center}
\caption{Messwerte der Aufgabe 3.6}
\end{table}

Im Versuch wurde der Lastwiderstand durch eine Widerstandsdekade eingestellt.
Der Laststrom wurde dann durch
$$ I_L = \frac{R_L}{U_a} $$
\noindent berechnet.

$$ \Delta U_a = U_{a_{max}} - U_{a_{min}} = (5.87 - 5.79) \si{\volt} = 0.08 \,\
\si{\volt}$$

\begin{center}
 \includegraphics[width=\textwidth]{2_5/2_5}
\end{center}

\end{document}
