Unterschieden wird in \emph{physikalische} Topologien (realer Aufbau des Netzwerkes, physische Verbindungen, etc.) und \emph{logische} Topologien (prinzipielle Verbindungen der Teilnehmer ohne Details über physische Lage etc.). Je nach Topologie gibt es unterschiedliche Ausfallsicherheiten.

\subsubsection{Linie}

\begin{figure}[H]
\centering
\resizebox{0.618\textwidth}{!}{\import{graphics/}{top_linie.pdf_tex}}
\end{figure}

\subsubsection{Bus}

\begin{figure}[H]
\centering
\resizebox{0.618\textwidth}{!}{\import{graphics/}{top_bus.pdf_tex}}
\end{figure}

\subsubsection{Ring}

\begin{figure}[H]
\centering
\resizebox{0.618\textwidth}{!}{\import{graphics/}{top_ring.pdf_tex}}
\end{figure}

\subsubsection{Stern}

\begin{figure}[H]
\centering
\resizebox{0.618\textwidth}{!}{\import{graphics/}{top_stern.pdf_tex}}
\end{figure}

\subsubsection{Vollvermaschung}

\begin{figure}[H]
\centering
\resizebox{0.618\textwidth}{!}{\import{graphics/}{top_vollvermaschung.pdf_tex}}
\end{figure}
