Im freien Raum breiten sich elektromagnetische Wellen mit der Lichgeschwindigkeit
\[c_0 \approx 3 \cdot 10^8\]
aus, welche die maximale Ausbreitungsgeschwindigkeit für Energie und Information
darstellt. In Leitungen ist die Ausbreitungsgeschwindigkeit $c$ von Spannungs- und
Stromwellen jedoch geringer. Dadurch erscheint die Wellenlänge $\lambda$ auf der Leitung
gegenüber der Wellenlänge im Vakuum $\lambda_0$ verkürzt. Das Verhältnis ist der Verkürzungsfaktor.
\[\frac{\lambda}{\lambda_0} = VK\]
Für die Wellenlänge auf der Leitung ergibt sich damit
\[\lambda = \frac{c}{f} = VK \cdot \frac{c_0}{f}\]

Der konkrete Wert des Verkürzungsfaktors ist von der Bauart der Leitung
abhängig. Für eine koaxiale Leitung ergibt sich beispielsweise
\[ {VK} = \frac{1}{\sqrt{\epsilon_r}}\]
Für ein Koaxialkabel mit einem Dielektrikum aus Polyethylen ($\epsilon_r =
2.25$) erhält man somit einen Verkürzungsfaktor von (näherungsweise) $VK = 0.667$.
