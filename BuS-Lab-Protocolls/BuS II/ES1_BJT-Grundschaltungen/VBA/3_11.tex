Man kann den Betrag der Stromverstärkung $\beta$ des Bipolartransistors über der
Frequenz (z.B. im Bode-Diagramm) auftragen. Der Einfluss der parasitären
Kapazitäten resultiert in einem tiefpassartigen Verlauf der Stromverstärkung.
Markante Punkte sind die Grenzfrequenz, bei welcher die Verstärkung um den
Faktor $\frac{1}{\sqrt{2}}$ (halbierte Leistung) fällt, sowie die
Transitfrequenz, bei welcher der Betrag der Verstärkung gleich eins ist.

Die Bandbreite beschreibt die Differenz zwischen der maximalen und der
minimalen Frequenz, die eine Dämpfung der Stromverstärkung um
$\frac{1}{\sqrt{2}}$ aufweisen (obere und untere Grenzfrequenz).